\documentclass[xcolor=dvipsnames]{beamer}

\usepackage[english]{babel}
\usepackage[latin1]{inputenc}
\usepackage[T1]{fontenc}

% --- Graphics inclusion.
\usepackage{graphicx}
%\usepackage{float}
%\usepackage{subfigure}
%\usepackage[all]{xy}

% --- Listing inclusion.
% \usepackage{dsfont}
% \usepackage{listings}
% \lstset{numbers=left, numberstyle=\tiny, numbersep=5pt}
% \lstset{language=C}

% --- Verbatim file inclusion.
%\usepackage{verbatim} % include entire files verbatim, \verbatiminput{...}

% --- Indent verbatim environments
\makeatletter \def\verbatim@startline{\verbatim@line{\leavevmode\kern20pt\relax}} \makeatother

% --- Color names.
%\usepackage{color}
%\usepackage[usenames,dvipsnames]{xcolor}
%\usepackage[svgnames]{xcolor}
% Define user colors using the RGB model
%\definecolor{darkgrey}{rgb}{0.8,0.8,0.8}
%\definecolor{lightgrey}{rgb}{0.95,0.95,0.95}
\definecolor{highlight}{named}{NavyBlue}

% --- Table mashups.
%\usepackage{colortbl}
%\usepackage{tabularx}
%\usepackage{longtable}
%\renewcommand*\arraystretch{2.0}

% --- Clubs and widows.
\clubpenalty = 10000
\widowpenalty = 10000
\displaywidowpenalty = 10000

% --- Vector fonts.
%\usepackage{mathptmx}
%\usepackage[scaled=.90]{helvet}
%\usepackage{courier}
%\usepackage{times} % True Type for normal text
%\usepackage{mathptmx} % True Type for maths

% --- Misc.
%\usepackage{trfrac}   % frac-like lines that are no fractions.
%\usepackage{moreverb} % More fun with {verbatim}
%\usepackage{amsfonts} % mathbb{N} and similar symbols
\usepackage{amsmath}  % \overset
\usepackage{amssymb}  % \ltimes
%\usepackage{pxfonts}  % \lJoin etc.; TypeSystems
%\setcounter{secnumdepth}{3}
%\setcounter{tocdepth}{3}

%% Drawing trees and other stuff
\usepackage{tikz}
\usetikzlibrary{trees,arrows}
\usepackage{amssymb}

\let\tikzsquare\square
\renewcommand{\square}{\ensuremath\tikzsquare}

\definecolor{cffffff}{RGB}{255,255,255}
\definecolor{c787aff}{RGB}{120,122,255}
\definecolor{cff7374}{RGB}{255,115,116}
\definecolor{c79ff79}{RGB}{121,255,121}
\definecolor{cff7374}{RGB}{255,115,116}
\definecolor{c7372ff}{RGB}{115,114,255}
\definecolor{c63ff63}{RGB}{99,255,99}
\definecolor{c6365ff}{RGB}{99,101,255}
\definecolor{darkgreen}{RGB}{0,162,0}
\definecolor{darkred}{RGB}{162,0,0}


% ----- BEAMER SPECIFIC -----


% --- Pretty much the usual beamer theme.
\usepackage{beamerthemeshadow}

% --- A title slide for every new section.
\AtBeginSection{
  \frame{
    \begin{block}{}
      \begin{Large}\color{highlight}\centerline{\insertsection}\end{Large}
    \end{block}
  }
}

% --- Remove the navigation bar from the bottom right.
\beamertemplatenavigationsymbolsempty

% --- Fade in upcoming bullet points.
\beamersetuncovermixins{\opaqueness<1>{25}}{\opaqueness<2->{15}}




\begin{document}

\title{Visigoth.}
\author{}
\date{10 January 2012}

\frame{\titlepage}
% \frame{\frametitle{Outline}\tableofcontents}




\section{Introduction}

\subsection{Motivation}
\frame{\frametitle{Motivation}
There is a whole class of barely explored graphs!
\pause

\vspace{0.5cm}
\emph{Example:} Graphs in social networks:
\pause

\begin{itemize}
  \item Facebook
  \item Twitter
  \item identi.ca
\end{itemize}
}

% FIXME: Not entirely happy with the slide split here.
%        Maybe we can write some more or put the points together?

\subsection{Small World Networks}
\frame{\frametitle{Small World Networks}
\begin{block}{Small World Networks}
  \emph{SWNs} are graphs that look like a natural web of friendships.
\end{block}
}

\subsection{Why SWNs?}
\frame{\frametitle{Why SWNs?}
SWNs are interesting because they show how people interact.

So let's have a look at them!
}

\subsection{Visigoth}
\frame{\frametitle{Visigoth}
Visigoth can generate or retrieve SWN graphs.

\vspace{0.5cm}
The user can analyse them both visually as well as statistically.

% FIXME: Screenshot here
\vspace{0.8cm}
(screenshot here)

\vspace{0.8cm}
Really, this is the cutting edge in SWN science!
}




\section{Maths}

% What is a graph
% Random graph vs. graphs closer to SWNs
% Three properties of SWNs

\subsection{Small World Networks}

\frame{\frametitle{Keeping it real}
Let's grow a population. They will make friends with each other.\\
\pause
\hspace{0.5cm} $\to$ annoying (crying babies!)\\
\hspace{0.5cm} $\to$ expensive (feeding them!)\\
\hspace{0.5cm} $\to$ maybe a little too slow...
\pause

\vspace{0.5cm}
So what can we do?
}

\frame{\frametitle{Generating Small World Networks}
Okay, let's generate SWNs mathematically then.

That should be a lot more convenient!
\pause

\vspace{0.5cm}
Nowadays, we generate networks of \emph{thousands} of nodes within seconds.

\pause
\hspace{0.5cm} $\to$ 10,000 nodes within 3 sec

\pause
\vspace{0.5 cm}
A \emph{network} is, in the mathematical sense, \pause a \emph{graph} which
consists of a set N of nodes and a set E of edges.

}

\frame{\frametitle{Small World Effect as a concept}
\emph{Six degrees of separation} 
\hspace{0.5cm} $\to$ the distance between any two individuals is on average $6$.

\begin{figure}[ht!]
  \centering
  \includegraphics[height=50mm,width=50mm]{sixdegrees.png}
  \caption{Social Network in Real Life}
\end{figure}

}

\frame{\frametitle{Small World Effect as a concept}
Frigyes Karinthy first introduced this idea in his 1929 short story \emph{Chains}.

\vspace{0.5cm}
\pause
Later on the small world experiment, performed in the 1960s by Stanley Milgram
and other researchers, formalised the aforementioned concept.

\vspace{0.5cm}
\pause
Results in a \emph{small world effect} 
\pause
\hspace{0.5cm} $\to$ the \emph{average distance} linking two nodes of the same
network can be orders of magnitude \emph{smaller than} the \emph{number of nodes} the 
network actually has.
}

\frame{\frametitle{Small World Network Characteristics}

Mathematically speaking, a small world network has the following properties:

\begin{itemize}
\pause
\item Low average path length (distance).
\pause
\item High clustering coefficient.
\pause
\item Scale-free degree distribution.
\end{itemize}
}

\frame{\frametitle{SWNs' characteristics insight}
The \emph{distance} between two nodes in a graph 
represents the minimum number of edges to be traversed in order to get from the
first node to the second.

\vspace{0.5cm}
\pause
\emph{Clustering} is the tendency of real networks to form \emph{cliques} 
(subset of nodes in which all nodes are connected to each other).

\vspace{0.5cm}
\pause
\emph{Scale-free} networks have a node degree distribution which follows a 
power-law:

\[ P(k) \sim k^{-\gamma} \hspace{1.2cm} \includegraphics[width=20mm, height=10mm]{angry.png} \] 
%add the angry face somewhere after the formula here..
}

\frame{\frametitle{More about scale-free networks}
\vspace{0.1cm}
Actually, this is not as hard as it seems! 
\hspace{0.3cm} \includegraphics[width=7mm,height=7mm]{happy.png}

\vspace{0.3cm}
Check this out!
\vspace{0.1cm}
\begin{figure}[ht!]
  \centering
  \includegraphics[height=40mm,width=60mm]{sfnet.png}
  \caption{Comparison between random and scale-free networks}
\end{figure}
}

\frame{\frametitle{Real World applications of SWNs}

\begin{columns}

\begin{column}[T]{0.3\textwidth}

\begin{figure}[ht!]
  \includegraphics[width=30mm,height=30mm]{roadmap.png}	
  \caption{London Road Map}
\end{figure}

\end{column}

\begin{column}[T]{0.3\textwidth}

\begin{figure}[ht!]
 \includegraphics[width=30mm,height=30mm]{neuralnets.png}
\caption{Neural networks}
\end{figure}

\end{column}

\begin{column}[T]{0.3\textwidth}

\begin{figure}[ht!]
 \includegraphics[width=30mm,height=30mm]{proteinstruct.png}
\caption{Protein Structure}
\end{figure}

\end{column}

\end{columns}

}

\frame{\frametitle{Moving forward...}
By integrating existing generation algorithms and visualisations into a single, 
easy-to-use interface, the user can make a head start into the large world of
small world networks.

\pause
\vspace{0.5cm}
So...let's check the \emph{algorithms} that help to the born of the small-world
networks' simulation...
}




\section{Engineering}

\subsection{Qt}


\subsection{C++}


\subsection{Graph drawing}

\frame{
  \frametitle{Graph drawing}

  Graph drawing is hard for humans... \pause and harder for computers:

  \begin{itemize}
    \pause
    \item Non-formal requirements (human taste)
    \pause
    \item Should adapt to a variety of different graphs
    \pause
    \item Should be fast
  \end{itemize}
}

\frame{
  \frametitle{Force-directed algorithms}

  For Visigoth, we use an optimized ``force directed'' algorithm.

  \pause

  Each node is a charged particle, and each edge is a rubber band.

  \pause

  We chose this kind of algorithm because:

  \begin{itemize}
    \item It gives good results for most graphs
    \pause
    \item It is easy to implement, few tens of lines of \emph{C++} (\emph{C++}!) in
      our case
    \pause
    \item We can show the stabilization in real time (bling)
  \end{itemize}

}

\subsection{FADE}

\frame{
  \frametitle{FADE}

  However, it also is.. \pause slow. Quite slow: $O(n^2)$, where $n$ is the
  number of nodes.

  \pause

  The solution? FADE, an algorithm that achieves speed through approximation.
}

\frame{
  \frametitle{TreeCodes}

  FADE works by storing the nodes of a \emph{TreeCode}, a data structure that
  stores the nodes recursively subdividing the space:

  \pause

  \begin{columns}
    \begin{column}{0.5\textwidth}
      \centering
      \input{quadtree-graph-pres.tex}
    \end{column}

    \pause

    \begin{column}{0.5\textwidth}
      \centering
      \begin{tikzpicture}[yscale=0.8, xscale=0.8, grow cyclic]
  \node {\square}
  child {
    node {\square}
    child {
      node {\square}
      child {node {\footnotesize{1}}}
    }
    child {
      node {\square}
      child {node {\footnotesize{2}}}
    }
  }
  child {
    node {\square}
    child {
      node {\square}
      child {
        node {\square}
        child { node{\footnotesize{3}} }
        child { node{\footnotesize{4}} }
      }
      child {
        node {\square}
        child {node{\footnotesize{5}}}
      }
    }
  }
  child {
    node {\square}
    child {
      node {\square}
      child {
        node {\square}
        child {node{\footnotesize{6}}}
        child {node{\footnotesize{7}}}
      }
    }
    child {
      node {\square}
      child {
        node {\square}
        child {node{\footnotesize{8}}}
      }
      child {
        node {\square}
        child {node{\footnotesize{9}}}
      }
    }
  }
  child {
    node {\square}
    child { node {\footnotesize{10}} }
  }
  ;
\end{tikzpicture}


    \end{column}
  \end{columns}
}

\frame{
  \frametitle{FADE}
  In our case, the \emph{TreeCode} subdivides the space recursively in 8 cubes.

  Once the \emph{TreeCode} is ready, when calculating the repulsion forces for
  each node:

  \begin{itemize}
    \pause
    \item Pick a cube in the \emph{TreeCode}
    \pause
    \item If the cube if ``far enough'', it will be treated as a single
      node, and the repulsion calculated based on how many nodes it contains
    \pause
    \item Otherwise, subdivide the cube, and try again with the 8 ``children''
      cubes.
  \end{itemize}
}

\subsection{C++}

\frame{
  \frametitle{C++}

  \emph{C++}, loved my many and feared by most...

}


\subsection{OpenGL}
\frame{\frametitle{OpenGL}
Visigoth is a lot of fun for mouse addicts as well.

\vspace{0.5cm}
We can...

\begin{itemize}
  \pause
  \item Pan left/right/up/down
  \pause
  \item Zoom in/out.
  \pause
  \item And do it all in 3D!
\end{itemize}
}




\section{DEMO}

\subsection{Demo}
\frame{\frametitle{Demo}
It's demo time!

\vspace{0.5cm}
{insert banana stuff here...}
}




\section{Conclusion}








%% BEAMER EXAMPLES


% \section{Example section 1}

\subsection{Subsec 1}

\frame{\frametitle{Slide Title. Pauses, bullet points}
Some text.
\pause

Some more text.
\pause

\begin{itemize}
  \item Bulletpoint 1.
  \pause
  \item Item 2 after a pause.
\end{itemize}
}


\subsection{Subsec 2}

\frame{\frametitle{Plain text}
Text....

\vspace{0.3cm}
More text....
}

\frame{\frametitle{Tables}
A table describing a Makefile
\vspace{0.5cm}

\begin{tabular}{ll}
  \texttt{make} & Compile\\
  \texttt{make run} & Run\\
  \texttt{make test} & Fail\\
  \texttt{make clean} & Clean up for distribution and inspection
\end{tabular}
}


\subsection{Huge tables}

\frame{\frametitle{Lotsa text}
This is a pretty full slide.
\vspace{0.5cm}

\begin{tabular}{ll}
  \texttt{Homer} & Family dad\\
  \texttt{Marge} & Family mom\\
  \\
  \texttt{Bart} & Son\\
  \texttt{Lisa} & Daughter\\
  \texttt{Maggie} & Teh baby\\
  \\
  \texttt{dog} & Not a cat.\\
  \texttt{cat} & Not a dog.\\
\end{tabular}

\vspace{0.3cm}
\begin{exampleblock}{How to seem surprised}
  \texttt{D'oh !}
\end{exampleblock}
}




% \section{More examples}

\frame{\frametitle{No subsection}
This slide is outside the normal subsectioning.
\pause

\vspace{0.5cm}
This can be used for section introductions.
}


\subsection{Going wild}

\frame{\frametitle{Mixing pauses and spaces}
Cat drank\\
\pause
\hspace{0.5cm} $\to$ Alice found.
\pause

\vspace{0.5cm}
Dog ate\\
\pause
\hspace{0.5cm} $\to$ Alice found something else.
\pause

\vspace{0.5cm}
So what's the point, anyway?
}


\subsection{Tables, pauses, spaces, (non-)linebreaks}

\frame{\frametitle{SD card: Development}
Pause before a table. NOT in the table!
Also, some vspace.
\pause

\vspace{0.5cm}
But no linebreak after this line and before the table:
\begin{tabular}{rl}
  123 & not 321\\
  456 & 789
\end{tabular}
\pause

\vspace{0.5cm}
Hope this helps.
\pause

\vspace{0.5cm}
One more line, because it fit on the slide.
}

\end{document}
