\documentclass[a4paper,11pt,titlepage]{article}

%%\usepackage[top=1in, bottom=1in, left=1in, right=1in]{geometry}

\usepackage{url}
\usepackage{hyperref}
\hypersetup{pdfborder=0 0 0}

\usepackage[osf]{libertine}

\newcommand{\mailto}[1]{\href{mailto://#1}{<#1>}}

\let\stdsection\section         % because LaTeX cannot handle
                                % recursive commands
\renewcommand{\section}{\newpage\stdsection}

\begin{document}
\title{\Huge Visigoth\\\Large Graph visualizations}
\author{
  Andreea-Ingrid Funie\\\mailto{aif109@doc.ic.ac.uk}\and
  Alexandru Scvor\c tov\\\mailto{as10109@doc.ic.ac.uk}\and
  Francesco Mazzoli\\\mailto{fm2209@doc.ic.ac.uk}\and
  Marc-David Haubenstock\\mailto{mh808@doc.ic.ac.uk}\and
  Maximilian Staudt\\\mailto{ms9109@doc.ic.ac.uk}
}
\date{January 2012}
\maketitle

\begin{abstract}
Visigoth is a tool to generate, analyse and visualise Small World
Networks.  It makes these kinds of networks accessible to anyone new
to their mathematical properties and assists in discovering the
various properties they exhibit by presenting the particular networks
generated by some of the currently published algorithms.
\end{abstract}

\tableofcontents

\section{Introduction}

\subsection{What are small world networks?}

Small World Networks derive their denomination from the well-known
Small World Theorem, which states that any two persons are related
through a chain of at most seven friends.

Indeed, Small Wold Networks are graphs resembling the connections
within human social networks. In other words, they are graphs in which
nodes represent people, and edges between nodes represent relations of
some sort.

For example, if someone were to draw every user of a social network
(e.g. Twitter) on a canvas and then connect each pair of them if they
are registered as “friends” on the platform, the resulting graph is a
Small World Network.  Over the course of time, several mathematical
algorithms that generate random Small World Networks have been
discovered. However, the resulting graphs have only lately begun to
resemble those that have grown naturally in the form of social
networks.

\subsection{Why Visigoth?}

Visigoth makes peeking into the current state-of-the-art of artificial
Small World Networks simple and fun. By integrating existing
generation algorithms into a single, easy-to-use interface, the user
can make a head start into the small world of Small World Networks,
analyse how these algorithms have evolved over time, see the effects
different algorithm parameters have on the resulting networks, and
compare them to naturally grown networks.

\end{document}
